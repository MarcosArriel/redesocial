\chapter{Descrição Geral}
A rede social tem como foco a integração entre alunos, professores, pesquisadores e egressos, 
podendo compartilhar algo para promover a curiosidade dos usuários e também uma forma de acompanhar um pesquisador, professor, aluno ou pesquisa em específico. 	

Observando a frequência em que as pessoas utilizam de redes sociais para se comunicar com outras pessoas, percebeu-se que com a criação de uma rede social focada na docência, extensão e pesquisa seria uma forma de facilitar a comunicação entre os envolvidos. 


\section{Perspectivas do produto}

\section{Funções do produto}

\section{Características dos usuários}
Os usuários serão estudantes de cursos de graduação, estudantes de pós-graduação, mestrado e doutorado, além de professores e pesquisadores, com o intuito de se socializarem, debater assuntos científicos como artigos de periódicos, estabelecer conexões que viabilizem iniciativas de pesquisa científica ou de produtos e maneira de disseminar o conhecimento científico de maneira mais democrática nos círculos sociais dos usuários.


\section{Restrições gerais}
O software será desenvolvido utilizando HTML 5 e CSS3 na construção do ambiente web, também será usada a linguagem de Programação Java devido ao grande volume de informações que serão geradas pelos usuários, o banco de dados escolhido foi o MySQL por ser open source além de ser melhor aceito pela equipe desenvolvedora.

\section{Suposições e dependências}
