\chapter{Projeto}

Este capitulo tem como foco a apresentação do fluxograma ilustrativo do sistema por meio do Diagrama Entidade e Relacionamento(DER), demostrar as trocas de informação entre operações pelos Diagramas de Sequencia e uma visão estática da estrutura física sobre a qual o software será implementado pelo Diagrama de Implantação.

\section{Diagrama Entidade e Relacionamento (DER)}

O Diagrama Entidade e Relacionamento (DER) apresenta de forma gráfica as tabelas do banco de dados, mostrando os tipos de dados que estarão armazenados, bem como os relacionamentos existentes entre as tabelas.

A figura mostra as tabelas do banco de dados, onde a tabela de usuários possue uma ligação de 1..n com a tabela cidades, onde uma cidade pode ter vários usuários, porém um usuário pode morar em apenas uma cidade.

A tabela cidades possue uma ligação de 1..n com a tabela estados, onde um estado pode ter várias cidades e uma cidade pode pertencer apenas a um estado.

A tabela estados possue uma ligação de 1..n com a tabela paises, onde um pais pode ter vários estados, porém um estado pode pertencer apenas a um pais. 

\figura{DER}{15}{../banco_de_dados/redesocial.png}{Diagrama Entidade e Relacionamento (DER).}


\section{Diagramas de Sequência}

\section{Diagrama de Implantação}
