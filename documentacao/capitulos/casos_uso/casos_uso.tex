\chapter{Casos de Uso}
%-----------------------------------------------------------
% Exemplo de Caso de uso
%-----------------------------------------------------------
\casoDeUso
%Identificador
<<<<<<< HEAD
=======
{UCx}
%Nome
{Manter Produtos}
%Ator Principal
{Gerente}
%Interessados
{
Interessados 1 Interessados 2
Interessados 1 Interessados 2
Interessados 1 Interessados 2
Interessados 1 Interessados 2
Interessados 1 Interessados 2
Interessados 1 Interessados 2
Interessados 1 Interessados 2
}
%Pré-Condições
{Estar Logado}
%Pós-condições
{Produto Cadastrado}
%Fluxo Básico
{
passo1
passo2
passo3
}
%Fluxo Alternativo
{
passo1
passo2
passo3
}
%Frequência de Ocorrencia
{Aproximadamente 3 vezes}
%Problemas em Aberto
{
Problemas que estão em aberto
}


%-----------------------------------------------------------
\casoDeUso
%Identificador
>>>>>>> 0edf1d3fac34f686cbcee7ecb8d59684d4d11ed9
{UC1}
%Nome
{Realizar cadastro}
%Ator Principal
{Usuário}
%Interessados
{
\begin{itemize}
	\item Usuário: deseja cadastrar se na rede social.	
\end{itemize}

}
%Pré-Condições
{Ter um e-mail válido.}
%Pós-condições
{Dados são enviados para o servidor.}
%Fluxo Básico
{
\begin{itemize}
	\item Usuário preenche os campos de nome, sobrenome , e-mail e data de nascimento.
	%\item Usuário aceita os termos de uso da rede social.
	%\item Usuário clica no botão \textit{“Criar conta”}.
	\begin{itemize}
		\item Usuário aceita os termos de uso da rede social.	
		\begin{itemize}
			\item Usuário clica no botão \textit{“Criar conta”}.
		\end{itemize}
	\end{itemize}
	
	
	\item Próxima etapa do cadastro é realizada
	\begin{itemize}
		%\item Próxima etapa do cadastro é realizada
		
		\item O usuário pode escolher sua senha de acordo com   os requisitos, como quantidade mínima de caracteres e seus tipos.
		\item O usuário escolhe a foto de seu perfil, de maneira opcional.
		\item O usuário pode escrever uma frase que aparecerá no seu perfil, de maneira opcional.
		
		\item O usuário pode anexar informações relacionadas a sua carreira acadêmica e profissional, de maneira opcional.
		
		\item O usuário pode anexar informações sobre cidades que morou, de maneira opcional.
		
		\item O usuário anexa informações sobre cidades que mora atualmente.
		
		
	\end{itemize}
\item Segunda fase do cadastro é concluída.
\begin{itemize}
			
			\item O sistema envia e-mail para o usuário para confirmação.
		\end{itemize}	
\end{itemize}
}
%Fluxo Alternativo
{
\begin{itemize}
	\item Usuário não preenche o campos de nome, e precisa ser notificado.

	\item Usuário não preenche o campos de sobrenome, e precisa ser notificado.

	\item Usuário não preenche o campos de celular e/ou e-mail, e precisa ser notificado.

	\item Usuário não aceita os termos de uso da rede social.

	\begin{itemize}
		\item Eventuais problemas no servidor podem impedir o envio de dados, sendo preciso notificar o usuário sobre.
	\end{itemize}
\end{itemize}
}
%Frequência de Ocorrencia
{Aproximadamente 3 vezes por minuto.}
%Problemas em Aberto
{

}

%-----------------------------------------------------------

\casoDeUso
%Identificador
{UC2}
%Nome
{Enviar confirmação de cadastro}
%Ator Principal
{Usuário}
%Interessados
{
\begin{itemize}
	\item Usuário: deseja receber o e-mail de confirmação.
\end{itemize}

}
%Pré-Condições
{Ter recebido todos os dados de forma válida}
%Pós-condições
{E-mail de confirmação é enviado}
%Fluxo Básico
{
\begin{itemize}
	\item E-mail é enviado com sucesso pelo servidor.
	\begin{itemize}
		\item E-mail é recebido com sucesso pelo usuário.
	\end{itemize}

\end{itemize}
}
%Fluxo Alternativo
{
\begin{itemize}
\item E-mail é enviado para a caixa de spam do usuário.
\end{itemize}
}
%Frequência de Ocorrencia
{Aproximadamente 3 vezes por  minuto.}
%Problemas em Aberto
{
 
}

%-----------------------------------------------------------

\casoDeUso
%Identificador
{UC3}
%Nome
{Confirmar recebimento de confirmação. Ativar conta}
%Ator Principal
{Usuário}
%Interessados
{
\begin{itemize}
	\item Usuário: deseja confirmar e-mail e ganhar acesso a seu perfil na rede social.
\end{itemize}

}
%Pré-Condições
{E-mail de confirmação precisa ter sido enviado para o usuário}
%Pós-condições
{A conta do usuário é ativa}
%Fluxo Básico
{
\begin{itemize}
	\item Usuário envia a confirmação.
	\begin{itemize}		
		\item Itens opcionais surgem durante a terceira etapa de cadastro para o usuário preencher.
		
		\item Conta é liberada para o uso do usuário.
	\end{itemize}	 
\end{itemize}
}
%Fluxo Alternativo
{
\begin{itemize}
	\item A confirmação não extá funcionando corretamente ou expirou.
	\begin{itemize}
		\item Usuário solicita o reenvio da confirmação.
	\end{itemize}
	\item Usuário deixa em branco itens opcionais de cadastro.
	\begin{itemize}
		\item Sistema informa que usuário não preencheu tais itens opcionais e que eles podem ser posteriormente editados.
	\end{itemize}
\end{itemize}
}
%Frequência de Ocorrencia
{Aproximadamente 3 vezes por minuto.}
%Problemas em Aberto
{

}

<<<<<<< HEAD

\casoDeUso
%Identificador
{UC4}
%Nome
{Enviar Post}
%Ator Principal
{Usuário}
%Interessados
{
\begin{itemize}
	\item Usuário: deseja realizar publicações em sua rede social.
\end{itemize}

}
%Pré-Condições
{Ter uma conta válida e ativa}
%Pós-condições
{Públicação é postada}
%Fluxo Básico
{
\begin{itemize}
\item Usuário acessa a sua timeline.
\item Usuário poderá postar foto, vídeo, link ou texto.
\item Apos inserir todas as informações o usuário aciona a opção de publicar post.
\end{itemize}
}
%Fluxo Alternativo
{
\begin{itemize}
\item Ocorreu algum erro ao publicar post.
\item É mostrado uma mensagem de erro ao inserir.
\item Usuário aciona a opção “Tentar Novamente”.
\end{itemize}
}
%Frequência de Ocorrencia
{Aproximadamente 50 vezes ao dia}
%Problemas em Aberto
{

}
=======
%-----------------------------------------------------------
\casoDeUso
%Identificador
{UC5}
%Nome
{Realizar Login}
%Ator Principal
{Usuário}
%Interessados
{
\begin{itemize}
	\item Usuário: deseja realizar login em sua conta.
	\item Servidor: deseja receber os dados do usuário para validar e realizar o login.
\end{itemize}

}
%Pré-Condições
{Possuir uma conta válida.}
%Pós-condições
{Login é realizado com sucesso.}
%Fluxo Básico
{
\begin{itemize}
\item Usuário acessa o site da redesocial.
\item Usuário digita seu usuário e senha, em seus respectivos campos.
\item Apos inserir as informações, o usuário aciona a opção de realizar login.
\end{itemize}
}
%Fluxo Alternativo
{
\begin{itemize}
\item Ocorreu algum erro ao realizar login.
\item É mostrado uma mensagem de erro ao tentar realizar o login.
\item Usuário não preenche os campos necessários.
\item Usuário não preenche os campos corretamente, de acordo com seus dados.
\item Problemas com o servidor que precisam ser notificados ao usuário.
\item Problemas com sua conexão com a internet.
\end{itemize}
}
%Frequência de Ocorrencia
{Aproximadamente 20 vezes ao dia.}
%Problemas em Aberto
{

}
