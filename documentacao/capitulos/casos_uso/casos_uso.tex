\chapter{Casos de Uso}

\casoDeUso
%Identificador
{UC1}
%Nome
{Realizar cadastro}
%Ator Principal
{Usuário}
%Interessados
{
\begin{itemize}
	\item Usuário: deseja cadastrar se na rede social.	
\end{itemize}

}
%Pré-Condições
{Ter um e-mail válido.}
%Pós-condições
{Dados são enviados para o servidor.}
%Fluxo Básico
{
\begin{itemize}
	\item Usuário preenche os campos de nome, sobrenome, e-mail e data de nascimento como mostra a Figura  \ref{figura:login}. %login	
	\item Usuário aceita os termos de uso da rede social.		
	\item Usuário clica no botão \textit{“Criar conta”}.	
	\item O sistema envia e-mail para o usuário para confirmação como mostra a Figura \ref{figura:2_sesso}.
	\item Após confirmar o e-mail, o usuário é redirecionado para a segunda etapa do cadastro.		
	\item O usuário preenche sua senha.
	\item O usuário preenche informações sobre local de trabalho, instituição de ensino, cidade natal, cidade atual, relacionamento e outras redes sociais como mostra a Figura \ref{figura:Cadastro_2_Passo}.
	\item Ao clicar no botão \textit{“Salvar e continuar”}, o usuário finaliza a segunda etapa de cadastro.	
	\item Usuário tem a opção de anexar uma foto para seu perfil usando sua webcam ou faz \textit{upload} de uma foto para o sistema como mostra a Figura \ref{figura:3_passo_cadastro_usuarios}. %falta a foto do protótipo da 3a etapa
	\item Ao clicar no botão \textit{“Salvar e continuar”}, o usuário finaliza a terceira etapa de cadastro.
	\item O usuário tem acesso a seu perfil.	
	
			
\end{itemize}
}
%Fluxo Alternativo
{
\begin{itemize}
	\item Usuário não preenche o campos de nome, sobrenome, celular e/ou e-mail, e precisa ser notificado.
	\item Usuário não aceita os termos de uso da rede social.
	\item Eventuais problemas no servidor podem impedir o envio de dados, sendo preciso notificar o usuário sobre.
	\item Usuário pode pedir o reenvio do e-mail de confirmação.
	\item Usuário pode não escolher uma foto para seu perfil.
	
\end{itemize}
}
%Frequência de Ocorrencia
{Aproximadamente 3 vezes por minuto.}
%Problemas em Aberto
{

}

%-----------------------------------------------------------

\casoDeUso
%Identificador
{UC2}
%Nome
{Enviar confirmação de cadastro}
%Ator Principal
{Usuário}
%Interessados
{
\begin{itemize}
	\item Usuário: deseja receber o e-mail de confirmação.
\end{itemize}

}
%Pré-Condições
{Ter recebido todos os dados de forma válida}
%Pós-condições
{E-mail de confirmação é enviado}
%Fluxo Básico
{
\begin{itemize}
	\item E-mail é enviado com sucesso pelo servidor.
	\item E-mail é recebido com sucesso pelo usuário.	
\end{itemize}
}
%Fluxo Alternativo
{
\begin{itemize}
\item E-mail é enviado para a caixa de spam do usuário.
\end{itemize}
}
%Frequência de Ocorrencia
{Aproximadamente 3 vezes por  minuto.}
%Problemas em Aberto
{
 
}

%-----------------------------------------------------------

\casoDeUso
%Identificador
{UC3}
%Nome
{Confirmar recebimento de confirmação. Ativar conta}
%Ator Principal
{Usuário}
%Interessados
{
\begin{itemize}
	\item Usuário: deseja confirmar e-mail e ganhar acesso a seu perfil na rede social.
\end{itemize}

}
%Pré-Condições
{E-mail de confirmação precisa ter sido enviado para o usuário}
%Pós-condições
{A conta do usuário é ativa}
%Fluxo Básico
{
\begin{itemize}
	\item Usuário clica no link de confirmação.
	\item Usuário é redirecionado para continuação do cadastro prosseguindo para a segunda etapa do cadastro.		
	\item O usuário preenche sua senha.
	\item O usuário anexa informações sobre local de trabalho, instituição de ensino, cidade natal, cidade atual, relacionamento e outras redes sociais como mostra a Figura \ref{figura:Cadastro_2_Passo}.
	\item Ao clicar no botão \textit{“Salvar e continuar”}, o usuário finaliza a segunda etapa de cadastro.	
	\item Usuário tem a opção de anexar uma foto para seu perfil usando sua webcam ou upando uma foto para o sistema.% como mostra a Figura \ref{figura:[...]}. %falta a foto do protótipo da 3a etapa
	\item Ao clicar no botão \textit{“Salvar e continuar”}, o usuário finaliza a terceira etapa de cadastro. 		
	\item Conta é liberada para o uso do usuário.		 
\end{itemize}
}
%Fluxo Alternativo
{
\begin{itemize}
	\item A confirmação não está funcionando corretamente ou expirou.
	\item Usuário solicita o reenvio da confirmação.
	
	\item Usuário deixa em branco itens opcionais de cadastro.
	\item Usuário pode clicar no botão \textit{“Pular”} ao invés de \textit{“Salvar e continuar”}.
	
	\item Sistema informa que usuário não preencheu tais itens opcionais e que eles podem ser posteriormente editados.
	
\end{itemize}
}
%Frequência de Ocorrencia
{Aproximadamente 3 vezes por minuto.}
%Problemas em Aberto
{

}

\casoDeUso
%Identificador
{UC4}
%Nome
{Enviar Post}
%Ator Principal
{Usuário}
%Interessados
{
\begin{itemize}
	\item Usuário: deseja realizar publicações em sua rede social.
\end{itemize}

}
%Pré-Condições
{Ter uma conta válida e ativa}
%Pós-condições
{Públicação é postada}
%Fluxo Básico
{
\begin{itemize}
\item Usuário acessa a sua timeline.
\item Usuário poderá postar foto, vídeo, link ou texto.
\item Apos inserir todas as informações o usuário aciona a opção de publicar post.
\end{itemize}
}
%Fluxo Alternativo
{
\begin{itemize}
\item Ocorreu algum erro ao publicar post.
\item É mostrado uma mensagem de erro ao inserir.
\item Usuário aciona a opção “Tentar Novamente”.
\end{itemize}
}
%Frequência de Ocorrencia
{Aproximadamente 50 vezes ao dia}
%Problemas em Aberto
{

}

%-----------------------------------------------------------
\casoDeUso
%Identificador
{UC5}
%Nome
{Realizar Login}
%Ator Principal
{Usuário}
%Interessados
{
\begin{itemize}
	\item Usuário: deseja realizar login em sua conta;
	\item Servidor: deseja receber os dados do usuário para validar e realizar o login.
\end{itemize}

}
%Pré-Condições
{Possuir uma conta válida.}
%Pós-condições
{Login é realizado com sucesso.}
%Fluxo Básico
{
\begin{itemize}
\item Usuário acessa o site da rede social;
\item O sistema exibe a página de login (Figura \ref{figura:login})
\item Usuário digita seu usuário e senha, em seus respectivos campos;
\item Após inserir as informações, o usuário aciona a opção de realizar login;
\item Servidor valida as informações digitadas pelo usuário e abre uma sessão para o usuário.
\end{itemize}
}
%Fluxo Alternativo
{
\begin{itemize}
\item Ocorreu algum erro ao realizar login;
\item É mostrado uma mensagem de erro ao tentar realizar login;
\item Usuário não preenche os campos necessários;
\item Usuário não preenche os campos corretamente, de acordo com seus dados;
\item Problemas com o servidor que precisam ser notificados ao usuário;
\item Problemas com sua conexão com a internet.
\end{itemize}
}
%Frequência de Ocorrencia
{Aproximadamente 20 vezes ao dia.}
%Problemas em Aberto
{

}

%-----------------------------------------------------------
\casoDeUso
%Identificador
{UC6}
%Nome
{Realizar comentarios}
%Ator Principal
{Usuário}
%Interessados
{
\begin{itemize}
	\item Usuário: Deseja comentar uma postagem.
	\item Autor da Postagem: Deseja ler os comentários publicados.
\end{itemize}

}
%Pré-Condições
{Estar validado no sistema}
%Pós-condições
{Cometário Publicado}
%Fluxo Básico
{
\begin{itemize}
	\item Usuário identifica uma postagem do seu interesse;
	\item Usuário digita o comentário conforme a figura \ref{figura:meus_posts};
	\item O usuário clica em "enviar" conforme a figura \ref{figura:meus_posts};
	\item Comentário é publicado.
\end{itemize}
}
%Fluxo Alternativo
{
\begin{itemize}
	\item Postagem foi deletada.
	\begin{itemize}
		\item Exibe uma mensagem informando que a postagem foi deletada.
		\item Volta para a tela anterior.
	\end{itemize}
	
	\item Falha na conexão com o servidor.
	
	\begin{itemize}
		\item Exibe uma mensagem informando que não foi possível conectar ao servidor;
		\item Volta para a tela anterior.
	\end{itemize}
	
	\item Conexão com a internet falhou.
	\begin{itemize}
		\item Exibe uma mensagem para verificar a conexão com a Internet;
		\item Volta para a tela anterior.
	\end{itemize}

	
	
\end{itemize}
}
%Frequência de Ocorrencia
{Aproximadamente 10 vezes ao dia.}
%Problemas em Aberto
{

}

%-----------------------------------------------------------
%-----------------------------------------------------------
\casoDeUso
%Identificador
{UC7}
%Nome
{Sair da rede social}
%Ator Principal
{Usuário}
%Interessados
{
\begin{itemize}
	\item Usuário: Deseja sair da rede social clicando no icone retangular proximo a foto como mostra a figura  \ref{figura:Layout_Fedd}.
\end{itemize}

}
%Pré-Condições
{Possuir uma conta válida e estar conectado (logado no sistema)}
%Pós-condições
{}
%Fluxo Básico
{
\begin{itemize}
	\item Usuário acessa o site da rede social e está conectado a sua conta.
	
\end{itemize}
}
%Fluxo Alternativo
{
\begin{itemize}
	\item Usuário desconectado.
	\item Exibir mensagem  ao usuário falando que ele está desconectado.
	\item Problemas de resposta do servidor.
	\item Exibir mensagem  ao usuário relatando problemas com resposta do servidor.
	\item Exibe uma mensagem informando que o usuário foi desconectado.
	\item Volta para a tela inicial do sistema de fazer login.
	\item Problemas com sua conexão com a internet.
\end{itemize}
}
%Frequência de Ocorrencia
{Aproximadamente 1 vez ao dia.}
%Problemas em Aberto
{

}
%-------------------------------------------------------------------------
\casoDeUso
%Identificador
{UC8}
%Nome
{Alterar dados de Perfil}
%Ator Principal
{Usuário}
%Interessados
{
\begin{itemize}
	\item Usuário: Deseja alterar seus dados de perfil.	
\end{itemize}
}
%Pré-Condições
{Estar conectado ao sistema e estar na página Perfil.}
%Pós-Condições
{Após alterações realizadas com sucesso, estes dados ser enviados ao servidor}
%Fluxo Básico
{
\begin{itemize}
	\item Usuário: Acessa o seu perfil como mostra a figura 4.
	\item Usuário: Clicar no menu Sobre.
	\item Usuário: Seleciona os campos que deseja alterar clicando nos ícones como mostra a figura 3.
	\item Usuário: Preenche os campos.
	\item Usuário: Confirma os campos ao clicar no botão "Confirmar".
\end{itemize}
}
%Fluxo Alternativo
{
\begin{itemize}
	\item Usuário desconectado.
	\item Problemas com resposta do servidor ao alterar o campo especificado.
	\item Exibir a mensagem de erro ao usuário.
	\item Usuário preencher o campo incorretamente ex:(Caracteres especiais, letras em campos de números, etc).
	\item Exibir mensagem de erro ao usuário sobre o preenchimento do campo.
	\item Exibir a forma correta de preencher.
	\item Usuário não confirmar a alteração.
	\item Informa o usuário se deseja salvar as alterações realizadas.
	\item Usuário inserir telefone inválido.
	\item Usuário inserir redes sociais inválidas.
	\item Sistema informar sobre os devidos erros de inserção.
	\item Usuário não estar conectado a internet.
	\item Perda das alterações realizadas, tendo, que refazer ao restabelecer a conexão com a internet.
	\item Usuário encerrar a sessão sem salvar as alterações.
	\item Usuário fechar o navegador.
\end{itemize}
}
%Frequência de Ocorrência
{Aproximadamente 3 vezes ao dia}
%Problemas em aberto
{

}
%--------------------------------------------------------------------------------------------------------
\casoDeUso
%Identificador
{UC9}
%Nome
{Pesquisar pessoas e grupos}
%Ator Principal
{Usuário}
%Interessados
{
\begin{itemize}
	\item Usuário: Deseja pesquisar uma pessoa ou grupo.	
\end{itemize}
}
%Pré-Condições
{Estar conectado ao sistema e estar na página Perfil.}
%Pós-Condições
{Após a pesquisa realizada com sucesso, as pessoas/grupos com o nome relacionado ao que foi pesquisado aparecem para que o usúario escolha.}
%Fluxo Básico
{
\begin{itemize}
	\item Usuário: Acessa o seu perfil como mostra a figura 4.
	\item Usuário: Clicar no campo de pesquisa.
	\item Usuário: Escrever o nome da pessoa/grupo que deseja pesquisar.
	\item Usuário: Selecionar a pessoa/grupo que aparecem com o nome relacionado ao nome que foi pesquisado.
	\item Usuário: O usúario e redirecionado para a pagina de perfil da pessoa ou para a pagina do grupo que foi selecionado.
\end{itemize}
}
%Fluxo Alternativo
{
\begin{itemize}
	\item Usuário desconectado.
	\item Problemas com resposta do servidor ao pesquisar uma pessoa/grupo.
	\item Exibir a mensagem de erro ao usuário.
	\item Usuário pesquisa um nome que não existe na rede social.
	\item Exibir mensagem de pessoa/grupo não encontrado.
	\item Usuário não estar conectado a internet.
\end{itemize}
}
%Frequência de Ocorrência
{Toda vez que o usúario requirir uma pesquisa de pessoa/grupo.}
%Problemas em aberto
{

}
%--------------------------------------------------------------------------------------------------------
\casoDeUso
%Identificador
{UC10}
%Nome
{Enviar mensagem no chat}
%Ator Principal
{Usuário}
%Interessados
{
\begin{itemize}
	\item Usuário: Enviar mensagem no chat para outros usuários.
	
\end{itemize}

}
%Pré-Condições
 {Possuir uma conta válida e estar conectado (logado no sistema}
%Pós-condições
{}
%Fluxo Básico
{
\begin{itemize}
	\item Usuário acessa o link mensagens.
	\item Usuário o sistema mostra outros usuários online ou offline (Figura \ref{figura:chat_ativos}).
	\item Usuário o ator seleciona um usuário online.
	\item Usuário o ator digita uma mensagem e clica no botão enviar (Figura \ref{figura:mensagens}).
\end{itemize}
}
%Fluxo Alternativo
{
\begin{itemize}
	\item Não haver conexão com a Internet .

\end{itemize}
}
%Frequência de Ocorrencia
{Diariamente.}
%Problemas em Aberto
{
}
%--------------------------------------------------------------------------------------------------------
\casoDeUso
%Identificador
{UC11}
%Nome
{Manter álbum}
%Ator Principal
{Usuário}
%Interessados
{
\begin{itemize}
	\item Usuário: seleciona a opção álbum.
	
\end{itemize}

}
%Pré-Condições
 {Possuir uma conta válida e estar conectado (logado no sistema}
%Pós-condições
{}
%Fluxo Básico
{
\begin{itemize}
	\item Usuário acessa ao álbum.
	\item Usuário o sistema mostra todos os álbum.
	\item Usuário o ator seleciona a foto ou álbum pode (visualizar, apaga, editar e postar fotos).
	\item Usuário o ator clicar na opção (criar álbum ou fotos) e o sitema redireciona para as fotos e seleciona a foto que será postada na rede social.
\end{itemize}
}
%Fluxo Alternativo
{
\begin{itemize}
	\item Não foi selecionado nenhum álbum .

\end{itemize}
}
%Frequência de Ocorrencia
{Duas vezes por semana.}
%Problemas em Aberto
{

}
%--------------------------------------------------------------------------------------------------------

