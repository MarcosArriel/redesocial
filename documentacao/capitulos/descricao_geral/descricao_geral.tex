\chapter{Descrição Geral}
A rede social tem como foco a integração entre alunos, professores, pesquisadores e egressos, 
podendo compartilhar algo para promover a curiosidade dos usuários e também uma forma de acompanhar um pesquisador, professor, aluno ou pesquisa em específico. 	

Observando a frequência em que as pessoas utilizam de redes sociais para se comunicar com outras pessoas, percebeu-se que com a criação de uma rede social focada na docência, extensão e pesquisa seria uma forma de facilitar a comunicação entre os envolvidos. 


\section{Perspectivas do produto}

O sistema em questão trata-se de uma rede social voltada para o contexto e agregação da comunidade científica. Espera-se adesão do público nessa plataforma, a qual poderá promover melhores relacionamentos interpessoais em um ambiente colaborativo, em prol da criação e incorporação de novos conceitos e valor. A mesma se propõe a contribuir para a geração e divulgação de pesquisas científicas, com discussões e conversas produtivas, indo além das mídias sociais comumente difundidas e podendo vir a compartilhar e ampliar publicações e destaques acadêmicos gerados no âmbito desta instituição e demais entidades colaborativas.

Ademais, o \textit{software} oferecerá a possibilidade de interação com outros sistemas, como Facebook e Google. O intuito é que possam fazer parte de um leque de auxílio para a criação de contas, facilitando a participação dos usuários na rede social descrita. Assim, qualquer um poderá entrar com sua conta Google, por exemplo, sem necessidade de preenchimento de todo o cadastro solicitado pela rede em foco.

Para tanto, o sistema deverá rodar em um servidor Java - \textit{GlassFish} (servidor \textit{opensource} de aplicação para Java EE), com no mínimo 8GB de memória, além de ambiente Linux. Ainda, verifica-se a necessidade de memórias principais e secundárias altas, como RAM de no mínimo 4,00 GB e HD (Disco rígido) de pelo menos 200/300 GB, além de processador \textit{multicore} eficiente, em um frequência com especificações mínimas de 3.00 GHz e 4 núcleos. 


\section{Funções do produto}

O sistema terá como funções como: cadastrar usuários, login, enviar solicitações de amizades, edição de perfil, Timeline, postagem de publicações e fotos, além de criar pesquisar, participar, convidar, sair e denunciar grupos, editar e gerenciar sua conta, envio de mensagens, alteração de status do bate papo e o cadastramento de artigos científicos.

\section{Características dos usuários}
Os usuários serão estudantes de cursos de graduação, estudantes de pós-graduação, mestrado e doutorado, além de professores e pesquisadores, com o intuito de se socializarem, debater assuntos científicos como artigos de periódicos, estabelecer conexões que viabilizem iniciativas de pesquisa científica ou de produtos e maneira de disseminar o conhecimento científico de maneira mais democrática nos círculos sociais dos usuários.


\section{Restrições gerais}
O software será desenvolvido utilizando HTML 5 e CSS3 na construção do ambiente web, também será usada a linguagem de Programação Java devido ao grande volume de informações que serão geradas pelos usuários, o banco de dados escolhido foi o MySQL por ser open source além de ser melhor aceito pela equipe desenvolvedora.

\section{Suposições e dependências}
O software exige que os seguintes programas previamente instalados no computador: MySQL.Também exige que a máquina do usuário tenha no mínimo processador dual core e 8 Gb de RAM para funcionar perfeitamente. Funcionará em computadores desktop e smartphones através dos seguintes navegadores: Mozilla Firefox e Google Chrome. Terá suporte de funcionamento no Windows e no Linux (especificamente a distribuição Ubuntu).