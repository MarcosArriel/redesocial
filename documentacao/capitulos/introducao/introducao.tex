\chapter{Introdução}

A rede social é um trabalho desenvolvido pelos alunos do curso de bacharelado em Sistemas de Informação sob orientação do professor Ronneesley Moura Teles.
O trabalho tem cunho didático e está voltado para o aprendizado de programação WEB e programação Orientada a Objetos.
Este documento descreve as informações básicas sobre a rede social, como seus requisitos e casos de uso.


\section{Escopo do produto}
A Rede Social idealizada pela equipe de desenvolvimento não tem um nome definido, sendo chamada apenas de "rede social".
O ambiente de interação social pode ser utilizados não só por docentes e dicentes, mas também por pesssoas que já tiveram vinculo com a instituição (egressos, desistentes e outros) ou até mesmo
pessoas que tem interesse na instituição (candidatos, pesquisadores, palestrantes e outros).

A rede social pode oferecer vários benefícios, como: melhor interação entre alunos e professores, amplificação da difusão do conhecimento e informações científicas.
Também pode ajudar na divulgação de trabalhos e eventos científicos de forma a contribuir e sempre agregar valor aos estudos.

A rede social deve oferecer postagens de vídeos, fotos, grupos de assuntos acadêmicos, postagens de artigos científicos, dissertações e teses de forma que sempre seja possível 
acontecer discussões sobre os assuntos postados. A rede social não tem a intenção de servir como um banco de teses/dissertações, mas sim promover o conhecimento científico e permitir
que pessoas possam tirar dúvidas interagindo através dos comentários de postagens. 
  

\section{Estrutura do documento}

O capítulo 2 apresenta a perspectiva, funções, características, restrições e suposições e suas pendências do produto tendo como base descrições menos detalhadas do produto já o capítulo 3 requisitos funcionais e não-funcionais descreve o que o software realiza e como realiza como desempenho, interface entre outros, em seguida o capítulo 4 protótipos é uma visão inicial de um sistema de software, onde possibilita demonstrar conceitos, experimentar opções de projeto, mostrando toda a interface inicial que talvez sofra alterações futuras e o capítulo 5 descrever como será o uso de uma funcionalidade de um sistema, demonstrando o fluxo correto para seu funcionamento como funcionalidades de alguns botões e níveis de acesso do usuário.