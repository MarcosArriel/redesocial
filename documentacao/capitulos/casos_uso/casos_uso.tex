\chapter{Casos de Uso}

\casoDeUso
%Identificador
{UC1}
%Nome
{Realizar cadastro}
%Ator Principal
{Usuário}
%Interessados
{
\begin{itemize}
	\item Usuário: deseja cadastrar se na rede social.	
\end{itemize}

}
%Pré-Condições
{Ter um e-mail válido.}
%Pós-condições
{Dados são enviados para o servidor.}
%Fluxo Básico
{
\begin{itemize}
	\item Usuário preenche os campos de nome, sobrenome, e-mail e data de nascimento como mostra a Figura  \ref{figura:login}. %login	
	\item Usuário aceita os termos de uso da rede social.		
	\item Usuário clica no botão \textit{“Criar conta”}.	
	\item O sistema envia e-mail para o usuário para confirmação como mostra a Figura \ref{figura:2_sesso}.
	\item Após confirmar o e-mail, o usuário é redirecionado para a segunda etapa do cadastro.		
	\item O usuário preenche sua senha.
	\item O usuário anexa informações sobre local de trabalho, instituição de ensino, cidade natal, cidade atual, relacionamento e outras redes sociais como mostra a Figura \ref{figura:Cadastro_2_Passo}.
	\item Ao clicar no botão \textit{“Salvar e continuar”}, o usuário finaliza a segunda etapa de cadastro.	
	\item Usuário tem a opção de anexar uma foto para seu perfil usando sua webcam ou upando uma foto para o sistema.% como mostra a Figura \ref{figura:[...]}. %falta a foto do protótipo da 3a etapa
	\item Ao clicar no botão \textit{“Salvar e continuar”}, o usuário finaliza a terceira etapa de cadastro.
			
	
			
\end{itemize}
}
%Fluxo Alternativo
{
\begin{itemize}
	\item Usuário não preenche o campos de nome, sobrenome, celular e/ou e-mail, e precisa ser notificado.
	\item Usuário não aceita os termos de uso da rede social.
	\item Eventuais problemas no servidor podem impedir o envio de dados, sendo preciso notificar o usuário sobre.
	\item Usuário pode pedir o reenvio do e-mail de confirmação.
	\item Usuário pode não escolher uma foto para seu perfil.
	
\end{itemize}
}
%Frequência de Ocorrencia
{Aproximadamente 3 vezes por minuto.}
%Problemas em Aberto
{

}

%-----------------------------------------------------------

\casoDeUso
%Identificador
{UC2}
%Nome
{Enviar confirmação de cadastro}
%Ator Principal
{Usuário}
%Interessados
{
\begin{itemize}
	\item Usuário: deseja receber o e-mail de confirmação.
\end{itemize}

}
%Pré-Condições
{Ter recebido todos os dados de forma válida}
%Pós-condições
{E-mail de confirmação é enviado}
%Fluxo Básico
{
\begin{itemize}
	\item E-mail é enviado com sucesso pelo servidor.
	\item E-mail é recebido com sucesso pelo usuário.	
\end{itemize}
}
%Fluxo Alternativo
{
\begin{itemize}
\item E-mail é enviado para a caixa de spam do usuário.
\end{itemize}
}
%Frequência de Ocorrencia
{Aproximadamente 3 vezes por  minuto.}
%Problemas em Aberto
{
 
}

%-----------------------------------------------------------

\casoDeUso
%Identificador
{UC3}
%Nome
{Confirmar recebimento de confirmação. Ativar conta}
%Ator Principal
{Usuário}
%Interessados
{
\begin{itemize}
	\item Usuário: deseja confirmar e-mail e ganhar acesso a seu perfil na rede social.
\end{itemize}

}
%Pré-Condições
{E-mail de confirmação precisa ter sido enviado para o usuário}
%Pós-condições
{A conta do usuário é ativa}
%Fluxo Básico
{
\begin{itemize}
	\item Usuário clica no link de confirmação.
	\item Usuário é redirecionado para continuação do cadastro prosseguindo para a segunda etapa do cadastro.		
	\item O usuário preenche sua senha.
	\item O usuário anexa informações sobre local de trabalho, instituição de ensino, cidade natal, cidade atual, relacionamento e outras redes sociais como mostra a Figura \ref{figura:Cadastro_2_Passo}.
	\item Ao clicar no botão \textit{“Salvar e continuar”}, o usuário finaliza a segunda etapa de cadastro.	
	\item Usuário tem a opção de anexar uma foto para seu perfil usando sua webcam ou upando uma foto para o sistema.% como mostra a Figura \ref{figura:[...]}. %falta a foto do protótipo da 3a etapa
	\item Ao clicar no botão \textit{“Salvar e continuar”}, o usuário finaliza a terceira etapa de cadastro. 		
	\item Conta é liberada para o uso do usuário.		 
\end{itemize}
}
%Fluxo Alternativo
{
\begin{itemize}
	\item A confirmação não está funcionando corretamente ou expirou.
	\item Usuário solicita o reenvio da confirmação.
	
	\item Usuário deixa em branco itens opcionais de cadastro.
	\item Usuário pode clicar no botão \textit{“Pular”} ao invés de \textit{“Salvar e continuar”}.
	
	\item Sistema informa que usuário não preencheu tais itens opcionais e que eles podem ser posteriormente editados.
	
\end{itemize}
}
%Frequência de Ocorrencia
{Aproximadamente 3 vezes por minuto.}
%Problemas em Aberto
{

}

\casoDeUso
%Identificador
{UC4}
%Nome
{Enviar Post}
%Ator Principal
{Usuário}
%Interessados
{
\begin{itemize}
	\item Usuário: deseja realizar publicações em sua rede social.
\end{itemize}

}
%Pré-Condições
{Ter uma conta válida e ativa}
%Pós-condições
{Públicação é postada}
%Fluxo Básico
{
\begin{itemize}
\item Usuário acessa a sua timeline.
\item Usuário poderá postar foto, vídeo, link ou texto.
\item Apos inserir todas as informações o usuário aciona a opção de publicar post.
\end{itemize}
}
%Fluxo Alternativo
{
\begin{itemize}
\item Ocorreu algum erro ao publicar post.
\item É mostrado uma mensagem de erro ao inserir.
\item Usuário aciona a opção “Tentar Novamente”.
\end{itemize}
}
%Frequência de Ocorrencia
{Aproximadamente 50 vezes ao dia}
%Problemas em Aberto
{

}

%-----------------------------------------------------------
\casoDeUso
%Identificador
{UC5}
%Nome
{Realizar Login}
%Ator Principal
{Usuário}
%Interessados
{
\begin{itemize}
	\item Usuário: deseja realizar login em sua conta.
	\item Servidor: deseja receber os dados do usuário para validar e realizar o login.
\end{itemize}

}
%Pré-Condições
{Possuir uma conta válida.}
%Pós-condições
{Login é realizado com sucesso.}
%Fluxo Básico
{
\begin{itemize}
\item Usuário acessa o site da redesocial.
\item Usuário digita seu usuário e senha, em seus respectivos campos.
\item Apos inserir as informações, o usuário aciona a opção de realizar login.
\item Servidor valida as informações digitadas pelo usuário e realiza o login do usuário.
\end{itemize}
}
%Fluxo Alternativo
{
\begin{itemize}
\item Ocorreu algum erro ao realizar login.
\item É mostrado uma mensagem de erro ao tentar realizar o login.
\item Usuário não preenche os campos necessários.
\item Usuário não preenche os campos corretamente, de acordo com seus dados.
\item Problemas com o servidor que precisam ser notificados ao usuário.
\item Problemas com sua conexão com a internet.
\end{itemize}
}
%Frequência de Ocorrencia
{Aproximadamente 20 vezes ao dia.}
%Problemas em Aberto
{

}

%-----------------------------------------------------------
\casoDeUso
%Identificador
{UC6}
%Nome
{Realizar comentarios}
%Ator Principal
{Usuário}
%Interessados
{
\begin{itemize}
	\item Usuário: Deseja comentar uma postagem.
	\item Autor da Postagem: Deseja ler os comentários publicados.
\end{itemize}

}
%Pré-Condições
{Possuir uma conta válida}
%Pós-condições
{}
%Fluxo Básico
{
\begin{itemize}
\item Usuário acessa o site da rede social.
\item Usuário faz login.
\item Usuário identifica uma postagem do seu interesse.
\item O usuário digita e publica seu comentário.
\end{itemize}
}
%Fluxo Alternativo
{
\begin{itemize}
\item Postagem foi deletada.
\item Problemas de resposta do servidor.
\item Exibe uma mensagem informando que a postagem foi deletada.
\item Volta para a tela anterior.
\item Problemas com sua conexão com a internet.
\end{itemize}
}
%Frequência de Ocorrencia
{Aproximadamente 10 vezes ao dia.}
%Problemas em Aberto
{

}

%-----------------------------------------------------------
%-----------------------------------------------------------
\casoDeUso
%Identificador
{UC7}
%Nome
{Sair da rede social}
%Ator Principal
{Usuário}
%Interessados
{
\begin{itemize}
	\item Usuário: Deseja sair da rede social.
\end{itemize}

}
%Pré-Condições
{Possuir uma conta válida e estar conectado (logado no sistema)}
%Pós-condições
{}
%Fluxo Básico
{
\begin{itemize}
	\item Usuário acessa o site da rede social.
	\item Usuário faz login.
	\item Usuário está conectado a sua conta.
\end{itemize}
}
%Fluxo Alternativo
{
\begin{itemize}
	\item Usuário desconectado.
	\item Problemas de resposta do servidor.
	\item Exibe uma mensagem informando que o usuário foi desconectado.
	\item Volta para a tela inicial do sistema de fazer login.
	\item Problemas com sua conexão com a internet.
\end{itemize}
}
%Frequência de Ocorrencia
{Aproximadamente 1 vez ao dia.}
%Problemas em Aberto
{

}