\chapter{Especificação de requisitos}
 	
 \section{Requisitos não funcionais}
 
 \requisitoNaoFuncional
 %Identificação do Requisito
 {RNF01}
 %Nome do Requisito
 {O sistema não deve permitir posts de usuários que configurem publicações ofensivas}
 %Fonte do Requisito
 {Anny Karoliny,Thalia Santana}
 %Local e/ou Reunião
 {Laboratório 2 do IF Goiano Ceres}
 %Data
 {26/07/2017}
 %Responsável pelo Requisito
 {Thalia Santos de Santana}
 %Especificação do Requisito
 {O sistema não deve permitir posts de usuários que configurem publicações ofensivas, desrespeitando os direitos humanos. Qualquer atividade e nomeação que infrinja o direito do outro poderá ser removida permanentemente.
 }
 
 \requisitoNaoFuncional
 %Identificação do Requisito
 {RNF02}
 %Nome do Requisito
 {O tempo de conclusão do software não poderá ultrapassar a data de 01/12/2017.}
 %Fonte do Requisito
 {Brener Gomes}
 %Local e/ou Reunião
 {Laboratório 2 do IF Goiano Ceres}
 %Data
 {26/07/2017}
 %Responsável pelo Requisito
 {Brener Gomes}
 %Especificação do Requisito
 {Caso o software não seja concluído até a data estipulada não será apresentado.
 }
 
 \requisitoNaoFuncional
 %Identificação do Requisito
 {RNF03}
 %Nome do Requisito
 {O sistema  deve ter suporte aos navegadores Google Chrome e Mozilla Firefox.}
 %Fonte do Requisito
 {Ronneesley Moura Teles}
 %Local e/ou Reunião
 {Laboratório 2 do IF Goiano Ceres}
 %Data
 {26/07/2017}
 %Responsável pelo Requisito
 {Anny Karoliny Moraes Ribeiro}
 %Especificação do Requisito
 {O sistema deve funcionar corretamente nos navegadores Google Chrome e Mozilla Firefox.
 }
 
 \requisitoNaoFuncional
 %Identificação do Requisito
 {RNF04}
 %Nome do Requisito
 {O sistema deve ter uma boa usabilidade.}
 %Fonte do Requisito
 {Anny Karoliny, Thalia Santana}
 %Local e/ou Reunião
 {Laboratório 2 do IF Goiano Ceres}
 %Data
 {26/07/2017}
 %Responsável pelo Requisito
 {Anny Karoliny Moraes Ribeiro}
 %Especificação do Requisito
 {O sistema deve ser de fácil aprendizagem, utilização e gerenciamento.
 }


 \section{Requisitos funcionais}
 	
 \requisitoFuncional
 %Identificação do Requisito
 {RF01}
 %Nome do Requisito
 {O sistema deve oferecer um cadastro de usuários}
 %Fonte do Requisito
 {Anny Karoliny,Thalia Santana}
 %Local e/ou Reunião
 {Laboratório 2 do IF Goiano Ceres}
 %Data
 {26/07/2017}
 %Responsável pelo Requisito
 {Anny Karoliny Moraes Ribeiro}
 %Especificação do Requisito
 {O sistema deverá conter um cadastro de usuários que deverá ser preenchido com os seguintes campos: nome, sobrenome, celular ou/e e mail e senha.
 }
 
 \requisitoFuncional
 %Identificação do Requisito
 {RF02}
 %Nome do Requisito
 {O sistema deve oferecer a opção de login}
 %Fonte do Requisito
 {Anny Karoliny,Thalia Santana}
 %Local e/ou Reunião
 {Laboratório 2 do IF Goiano Ceres}
 %Data
 {26/07/2017}
 %Responsável pelo Requisito
 {Anny Karoliny Moraes Ribeiro}
 %Especificação do Requisito
 {O usuário deverá realizar login em sua conta digitando seu e mail e sua senha cadastrados anteriormente.
 }
 
 \requisitoFuncional
 %Identificação do Requisito
 {RF03}
 %Nome do Requisito
 {O sistema deverá dar a opção de editar o perfil do usuário}
 %Fonte do Requisito
 {Anny Karoliny,Thalia Santana}
 %Local e/ou Reunião
 {Laboratório 2 do IF Goiano Ceres}
 %Data
 {26/07/2017}
 %Responsável pelo Requisito
 {Anny Karoliny Moraes Ribeiro}
 %Especificação do Requisito
 {O usuário poderá através desta opção editar  foto de perfil, foto de capa, biografia, local de trabalho, educação, cidade atual, cidade natal e status de relacionamento.
 }
 
 \requisitoFuncional
 %Identificação do Requisito
 {RF04}
 %Nome do Requisito
 {O software deverá conter uma Timeline}
 %Fonte do Requisito
 {Brener Gomes}
 %Local e/ou Reunião
 {Laboratório 2 do IF Goiano Ceres}
 %Data
 {26/07/2017}
 %Responsável pelo Requisito
 {Brener Gomes}
 %Especificação do Requisito
 {Na timeline irá visualizar as postagens mais recentes sendo mensagens, fotos e/ou vídeos sendo mensagens, fotos e/ou vídeos criados por amigos, grupos, páginas e outras do usuário, haverá um campo de criar publicação onde o usuário poderá descrever uma mensagem e/ou foto, vídeos, gifs, figuras, fazer marcação de amigos, locais em suas publicações, seguindo de propagandas de páginas e aplicativos podendo ser alterada ou excluída após sua criação. Ao descer a timeline irá demonstrando mais publicações, ou seja, não haverá divisão por páginas. O usuário terá acesso para comentar e compartilhar publicações de outros usuários.
 }
 
 \requisitoFuncional
 %Identificação do Requisito
 {RF05}
 %Nome do Requisito
 {O software disponibilizará um Post}
 %Fonte do Requisito
 {Brener Gomes}
 %Local e/ou Reunião
 {Laboratório 2 do IF Goiano Ceres}
 %Data
 {26/07/2017}
 %Responsável pelo Requisito
 {Brener Gomes}
 %Especificação do Requisito
 {Nesta seção o usuário terá disponível um campo onde poderá criar, editar ou excluir sua publicação, podendo inserir caracteres, fotos, vídeos, figuras ou gifs que tenham as extensões suportadas pelo software, o usuário poderá também restringir qual público poderá vê la. Ao finalizar a publicação será automaticamente postada em sua timeline, assim, sendo visíveis aos seus amigos.
 }
 
 \requisitoFuncional
 %Identificação do Requisito
 {RF06}
 %Nome do Requisito
 {O software disponibilizará um Álbum de Fotos.}
 %Fonte do Requisito
 {Brener Gomes}
 %Local e/ou Reunião
 {Laboratório 2 do IF Goiano Ceres}
 %Data
 {26/07/2017}
 %Responsável pelo Requisito
 {Brener Gomes}
 %Especificação do Requisito
 {O álbum de fotos será demonstrado em forma de grade. O usuário cadastrado poderá editar, criar ou apagar o álbum. O álbum de fotos só aceitará fotos com as extensões .jpg, .png, .jpeg. O álbum será composto pelos seguintes campos que serão armazenados no banco de dados:
  Nome do Álbum;
  Local;
  Descrição do Álbum;
  Amigos.
 Após as fotos serem adicionadas e carregadas com êxito o usuário poderá adicionar legendas em cada foto e marcar amigos.
 }
 
 \requisitoFuncional
 %Identificação do Requisito
 {RF07}
 %Nome do Requisito
 {O sistema deve permitir que o usuário crie grupos}
 %Fonte do Requisito
 {Anny Karoliny,Thalia Santana}
 %Local e/ou Reunião
 {Laboratório 2 do IF Goiano Ceres}
 %Data
 {26/07/2017}
 %Responsável pelo Requisito
 {Thalia Santos de Santana}
 %Especificação do Requisito
 {O usuário poderá criar um grupo dentro da rede social e este exibirá três opções de privacidade: público, fechado ou secreto. No público, todos podem ver o grupo, seus membros e publicações. Em fechado, pode se vê lo e os membros, mas não as publicações. Já em grupos secretos, só membros têm acesso aos mesmos. Os grupos ainda poderão abarcar diferentes temáticas, desde por exemplo, compra e vendas até entretenimento e esportes. Todo grupo deve também ter um nome e uma descrição do mesmo. O usuário ao criar um grupo passa a ser o administrador deste.
 }
 
 \requisitoFuncional
 %Identificação do Requisito
 {RF08}
 %Nome do Requisito
 {O sistema deve permitir que o usuário pesquise grupos}
 %Fonte do Requisito
 {Anny Karoliny,Thalia Santana}
 %Local e/ou Reunião
 {Laboratório 2 do IF Goiano Ceres}
 %Data
 {26/07/2017}
 %Responsável pelo Requisito
 {Thalia Santos de Santana}
 %Especificação do Requisito
 {O usuário poderá pesquisar grupos aos quais tenha interesse de participar. Ao buscar por estes, os grupos serão apresentados de acordo com sua categoria e temática, facilitando a busca do usuário. Só poderão ser exibidos grupos públicos e fechados.
 }
 
 \requisitoFuncional
 %Identificação do Requisito
 {RF09}
 %Nome do Requisito
 {O sistema deve permitir que o usuário participe de grupos}
 %Fonte do Requisito
 {Anny Karoliny,Thalia Santana}
 %Local e/ou Reunião
 {Laboratório 2 do IF Goiano Ceres}
 %Data
 {26/07/2017}
 %Responsável pelo Requisito
 {Thalia Santos de Santana}
 %Especificação do Requisito
 {O usuário poderá participar de grupos de seu interesse. Para os grupos públicos, haverá a opção de participar e a mesma,  ocorrerá de modo instantâneo. Em fechados, ao requerer a participação, uma solicitação será enviada aos administradores e moderadores deste. Apenas se a solicitação for aceita é que o usuário se torna oficialmente membro do grupo. Caso o grupo seja secreto, só se poderá participar caso um membro efetivo o adicione.  Em grupos, um membro pode comentar e fazer publicações, além de compartilhar o grupo (menos se ele for secreto).
 }
 
 \requisitoFuncional
 %Identificação do Requisito
 {RF10}
 %Nome do Requisito
 {O sistema deve permitir que o usuário saia de grupos}
 %Fonte do Requisito
 {Anny Karoliny,Thalia Santana}
 %Local e/ou Reunião
 {Laboratório 2 do IF Goiano Ceres}
 %Data
 {26/07/2017}
 %Responsável pelo Requisito
 {Thalia Santos de Santana}
 %Especificação do Requisito
 {O usuário poderá sair de grupos que desejar. Ao realizar essa atividade, o sistema deve proporcionar também a opção de que o usuário não possa mais ser adicionado de volta ao grupo em questão.
 }
 
 \requisitoFuncional
 %Identificação do Requisito
 {RF11}
 %Nome do Requisito
 {O sistema deve permitir que o usuário denuncie grupos}
 %Fonte do Requisito
 {Anny Karoliny,Thalia Santana}
 %Local e/ou Reunião
 {Laboratório 2 do IF Goiano Ceres}
 %Data
 {26/07/2017}
 %Responsável pelo Requisito
 {Thalia Santos de Santana}
 %Especificação do Requisito
 {O usuário pode denunciar o grupo quando o considerar ofensivo. Para tal ação, o motivo da denúncia deve ser informado para que a equipe de manutenção averigue a situação.
 }
 
 \requisitoFuncional
 %Identificação do Requisito
 {RF12}
 %Nome do Requisito
 {O sistema deve permitir que o usuário edite suas configurações gerais de conta}
 %Fonte do Requisito
 {Thalia Santana}
 %Local e/ou Reunião
 {Laboratório 2 do IF Goiano Ceres}
 %Data
 {26/07/2017}
 %Responsável pelo Requisito
 {Thalia Santos de Santana}
 %Especificação do Requisito
 {O usuário pode alterar suas configurações gerais de conta. São passíveis de edição os seguintes campos: Nome, Nome de usuário, Contato,  Senha.
 }
 
 \requisitoFuncional
 %Identificação do Requisito
 {RF13}
 %Nome do Requisito
 {O sistema deve permitir que o usuário gerencie sua conta}
 %Fonte do Requisito
 {Thalia Santana}
 %Local e/ou Reunião
 {Laboratório 2 do IF Goiano Ceres}
 %Data
 {26/07/2017}
 %Responsável pelo Requisito
 {Thalia Santos de Santana}
 %Especificação do Requisito
 {O usuário poderá gerenciar sua conta. A atividade pode ser dividida em 2 itens: desativar conta e excluir conta. Na primeira, o usuário pode voltar sempre que desejar, fazendo o login novamente, porém não será possível encontrar seu perfil nesse período. Já na exclusão, é um processo permanente,  o qual não é possível recuperar a mesma conta. Em ambos, é necessário informar o motivo para sua execução e realizar uma verificação de segurança com  senha e captcha.  
 }
 
 \requisitoFuncional
 %Identificação do Requisito
 {RF14}
 %Nome do Requisito
 {O sistema deve permitir que o usuário convide pessoas a participarem de grupos}
 %Fonte do Requisito
 {Thalia Santana}
 %Local e/ou Reunião
 {Laboratório 2 do IF Goiano Ceres}
 %Data
 {26/07/2017}
 %Responsável pelo Requisito
 {Thalia Santos de Santana}
 %Especificação do Requisito
 {O usuário poderá convidar pessoas (amigos) para participarem de um grupo em qual é participante ou mesmo, administrador.
 }
 
 \requisitoFuncional
 %Identificação do Requisito
 {RF15}
 %Nome do Requisito
 {O sistema deve permitir que o usuário envie solicitações de amizade para outros usuários da rede social}
 %Fonte do Requisito
 {Thalia Santana}
 %Local e/ou Reunião
 {Laboratório 2 do IF Goiano Ceres}
 %Data
 {26/07/2017}
 %Responsável pelo Requisito
 {Thalia Santos de Santana}
 %Especificação do Requisito
 {O usuário poderá enviar solicitações de amizade para demais usuários. O convite será enviado e pode ser aceito ou recusado.  Em caso de aceite, os usuários passam a estar ligados e podem ver e comentar as publicações um do outro, realizar marcações, enviar mensagens e postar na linha do tempo do amigo.
 }
 
 \requisitoFuncional
 %Identificação do Requisito
 {RF16}
 %Nome do Requisito
 {O sistema deve permitir que o  usuário envie mensagens}
 %Fonte do Requisito
 {Thalia Santana}
 %Local e/ou Reunião
 {Laboratório 2 do IF Goiano Ceres}
 %Data
 {26/07/2017}
 %Responsável pelo Requisito
 {Thalia Santos de Santana}
 %Especificação do Requisito
 {O usuário pode mandar mensagens para usuários da rede social. Em caso de amigos, as mensagens chegam no bate papo automaticamente. Sem a amizade, é necessário que o receptor aprove a mensagem enviada.
 }
 
 \requisitoFuncional{RF17}
 %Nome do Requisito
 {O sistema deve permitir que o usuário altere seu status no bate papo}
 %Fonte do Requisito
 {Anny Karoliny, Thalia Santana}
 %Local e/ou Reunião
 {Laboratório 2 do IF Goiano Ceres}
 %Data
 {26/07/2017}
 %Responsável pelo Requisito
 {Thalia Santos de Santana}
 %Especificação do Requisito
 { O usuário poderá alterar seu status no bate papo. Inicialmente, todo usuário está online, mas será possível desativar o mesmo.
 }	
 


    \requisitoFuncional{RF18}
 %Nome do Requisito
 {O sistema deve fornecer um cadastro de artigos científicos}
 %Fonte do Requisito
 {Andrey Silva Ribeiro}
 %Local e/ou Reunião
 {Laboratório 2 do IF Goiano Ceres}
 %Data
 {11/08/2017}
 %Responsável pelo Requisito
 {Andrey Silva Ribeiro}
 %Especificação do Requisito
 { O cadastro de artigos científicos no sistema somente será realizado com o aceite das condições de publicação por parte dos autores. Dada a permissão, os autores devem preencher o idioma, nome da revista, ISSN (International Standard Serial Number), seus nomes, data, área do conhecimento, título, resumo do artigo, url e arquivo.pdf do artigo como mostra a figura \ref{figura:cadastro_artigos}. Assim como a URL do artigo e o arquivo físico do artigo em PDF.
 }